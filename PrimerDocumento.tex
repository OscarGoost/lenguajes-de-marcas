\documentclass[10pt,a4paper]{report}
\usepackage[utf8]{inputenc}
\usepackage{amsmath}
\usepackage{amsfonts}
\usepackage{amssymb}
\begin{document}

\pagebreak
\tableofcontents
\pagebreak

\chapter{Modo texto}

{\underline Hola} {\it que} ta{\bf L}oco \\
\\
{\bf negrita} \\
{\it cursiva} \\
\underline{subrayado}

\section{Tamaños de letra}

\subsection{Tamaños pequeños}

{\tiny pequeño} \\
{\normalsize normal} \\

\subsection{Tamaños grandes}

{\large grande} \\
{\LARGE grandisimo} \\
{\huge enorme} \\

\chapter{Modo matemático}

\section{Modo en linea}

Voy a meter un exponente $ a^{\frac{8}{100}} $ y otro $ a^{n^{m^p}} $ \\


{\bf Ahora integrales:} \\

$ \int_{0}^{10} \cos^2 x dx $ \\

Una linea $ \int_{0}^{10} \frac{3x^2+x}{5x^2} dx $ de integrales

Expresión matemática en linea aparte:
$$ \int_{0}^{10} \frac{3x^2+x}{5x^2} x dx $$

Expresión matemática más grande:
$$ \int_{0}^{10} \frac{\frac{3x^2} x dx}{\frac{7x^3+x^2}{9x^2+6}} $$

una llave horizontal
$$ \underbrace{
		\int_{0}^{10}
			\frac{3x^2+x}{5x^2} dx
		+
		\int_{0}^{10}
			\frac{3x^2}{5x^2}
}_{x^4}
$$

$$\underbrace{
	\overbrace{
		\int_{0}^{10}
			\frac{3x^2+x}{5x^2} dx
	}
	^{x^{12}}
	+
	\overbrace{
		\int_{0}^{10}
			\frac{3x^2}{5x^2}
	}^{x^6}
}_{x_4}
$$

\chapter{Tablas}

Una tabla de ejemplo: \\

\begin{tabular}{|l|l|c|c|}
  \hline Antonio 	& Ruiz	& 8.6 & 4.9 \\
  \hline Eva			& López & 4.6 & 7.3 \\
  \hline Ana			& Eva	& 7.5 & 6.8 \\
  \hline
  
\end{tabular} \\ \\

\begin{tabular}{|c|c|c|c|c|}
  \hline Lunes 	& Martes	& Miércoles	& Jueves	& Viernes \\
  \hline	 3.5		& 7.4	& 2.6		& 2.7	& 7.5 \\
  \hline 2.4		& 4.6	& 6.4		& 7.8	& 8.9 \\
  \hline 1.7		& 8.6	& 5.4		& 3.5	& 2.3 \\
  \hline

\end{tabular}

\end{document}